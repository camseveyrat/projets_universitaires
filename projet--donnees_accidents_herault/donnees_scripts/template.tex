\documentclass[mstat,12pt]{unswthesis}

$if(highlighting-macros)$
$highlighting-macros$
$endif$


%%%%%%%%%%%%%%%%%%%%%%%%%%%%%%%%%%%%%%%%%%%%%%%%%%%%%%%%%%%%%%%%%%
% 
% OK...Now we get to some actual input.  The first part sets up
% the title etc that will appear on the front page
%
%%%%%%%%%%%%%%%%%%%%%%%%%%%%%%%%%%%%%%%%%%%%%%%%%%%%%%%%%%%%%%%%%

\title{Projet réalisé par\\[0.5cm] l'équipe $team$ du groupe de $groupeTD$ \\[3cm]$title$}

\authornameonly{$for(author)$$author$ $endfor$}

\author{\Authornameonly}

\copyrightfalse
\figurespagefalse
\tablespagefalse

%%%%%%%%%%%%%%%%%%%%%%%%%%%%%%%%%%%%%%%%%%%%%%%%%%%%%%%%%%%%%%%%%
%
%  And now the document begins
%  The \beforepreface and \afterpreface commands puts the
%  contents page etc in
%
%%%%%%%%%%%%%%%%%%%%%%%%%%%%%%%%%%%%%%%%%%%%%%%%%%%%%%%%%%%%%%%%%%


\input{header.tex}

\renewcommand{\contentsname}{Table des matières}

\renewcommand{\chaptername}{Chapitre}




\begin{document}

\beforepreface

%\afterpage{\blankpage}

% plagiarism

\prefacesection{Déclaration de non plagiat}


\vskip 2pc \noindent Nous déclarons que ce rapport est le fruit de notre seul travail, à part lorsque cela est indiqué  explicitement. 

\vskip 2pc  \noindent Nous acceptons que la personne évaluant ce rapport puisse, pour les besoins de cette évaluation:
\begin{itemize}
\item la reproduire et en fournir une copie à un autre membre de l'université; et/ou,
\item en communiquer une copie à un service en ligne de détection de plagiat (qui pourra en retenir une copie pour les besoins d'évaluation future).
\end{itemize}

\vskip 2pc \noindent Nous certifions que nous avons lu et compris les règles ci-dessus.\vspace{24pt}

\vskip 2pc \noindent En signant cette déclaration, nous acceptons ce qui précède.
\vskip 2pc \noindent
$for(author)$
Signature: \rule{7cm}{0.25pt} \hfill Date: \rule{4cm}{0.25pt} \\[1cm]
$endfor$
\vskip 1pc

%\textcolor{red}{Mettre à jour la date.}

{\bigskip\bigskip\bigskip\noindent} \today
%\afterpage{\blankpage}

% Acknowledgements are optional


\prefacesection{Remerciements}

{\bigskip}$for(Acknowledgements)$$Acknowledgements$\\[1cm] $endfor$

{\bigskip\bigskip\bigskip\noindent} \today

%\afterpage{\blankpage}

% Abstract

\prefacesection{Résumé}

  L’analyse des accidents de la route constitue un enjeu majeur pour orienter les politiques de sécurité et de prévention. 

\vspace{1em}

Ce projet porte sur les accidents survenus dans le département de l’Hérault en 2023, avec pour objectif d’identifier les facteurs associés à la gravité des blessures.

\vspace{1em}

L’étude s’appuie sur l’exploitation de données ouvertes, nettoyées et croisées entre plusieurs sources (usagers, véhicules, lieux et caractéristiques liés accidents). Trois grandes familles de facteurs ont été examinées : humains (âge, sexe, équipement), environnementaux (météo, luminosité) et liés à l’infrastructure routière (type de route, vitesse maximale autorisée, type de collision).

\vspace{1em}

Certains facteurs se sont révélés être très influents sur la gravité des accidents, notamment la plupart des facteurs liés à la route. Mais à l'inverse, d'autres caractéristiques semblent moins discriminantes sur le plan statistique.

\vspace{1em}

Ces observations permettent de formuler des recommandations concrètes en matière d’aménagement, de réglementation et de sensibilisation des usagers.

%\afterpage{\blankpage}


\afterpreface





%%%%%%%%%%%%%%%%%%%%%%%%%%%%%%%%%%%%%%%%%%%%%%%%%%%%%%%%%%%%%%%%%%
%
% Now we can start on the first chapter
% Within chapters we have sections, subsections and so forth
%
%%%%%%%%%%%%%%%%%%%%%%%%%%%%%%%%%%%%%%%%%%%%%%%%%%%%%%%%%%%%%%%%%%



%%%%%%%%%%%%%%%%%%%%%%%%%%%%%%%%%%%%%

%\afterpage{\blankpage}


$body$







\end{document}


